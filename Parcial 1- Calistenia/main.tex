\documentclass{article}



\begin{document}

\begin{titlepage}
    \begin{center}
        \vspace*{1cm}
            
        \Huge
        \textbf{Parcial 1-calistenia}
            
        \vspace{0.5cm}
        \LARGE
        
            
        \vspace{1.5cm}
            
        \textbf{Miguel Ángel Restrepo Rueda}
            
        \vfill
            
        \vspace{0.8cm}
            
        \Large
        Departamento de Ingeniería Electrónica y Telecomunicaciones\\
        Universidad de Antioquia\\
        Medellín\\
        Marzo de 2021
            
    \end{center}
\end{titlepage}

\tableofcontents
\newpage
\section{Pasos a seguir}\label{Pasos a seguir} 

 1.	Poner una toalla sobre la mesa, extenderla y dejarla totalmente extendida sin ninguna arruga durante todo el experimento sobre la mesa, puede ser una toalla de manos o una toalla para el cuerpo (La que se usa al secarse el cuerpo luego de bañarse), el material de la toalla debe ser de 100% algodón.\newline \newline

 2.	Obtener dos tarjetas de igual tamaño, tarjetas hechas de plástico (por ejemplo, los documentos de identificación). \newline \newline

 3.	Cabe resaltar que de ahora en adelante todo el proceso se hace con una sola mano. \newline \newline

 4.	Apilar las dos tarjetas con una mano una encima de la otra, de manera que las tarjetas queden traslapadas pareciendo solo una tarjeta. \newline \newline

 5.	Coger con una mano ambas tarjetas apiladas, mantenerlas traslapadas como se dijo en el paso anterior.\newline \newline

 6.	Poner las tarjetas de manera vertical sobre la toalla, manteniéndolas traslapadas como se dijo en el paso 4 y sostenerlas verticales.  \newline \newline

 7.	Separar suavemente ambas tarjetas haciendo que una quede recostada en la otra, hasta que las tarjetas parezca que se quedan inmóviles una sobre la otra. \newline \newline

 8.	En el caso de que se caigan las tarjetas repetir los pasos desde el paso 4 hasta el 7.\newline \newline

 9.	En el caso de que las tarjetas se queden sostenidas una sobre la otra, felicidades, ya acabó con el experimento. \newline \newline

\newpage

\section{Link del video} \label{Link del video}

Link del video en YouTube: https://youtu.be/Rsgq0qYmwJo 
 \newpage
 




\end{document}